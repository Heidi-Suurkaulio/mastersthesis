\section{Conclusion}
Network analysis can reveal new patterns on the family networks of the Swedish councillors of the Realm, but in this case, it mostly seems to support the already known. The graphs depictin the councillors' family ties in the early modern period are highly linked on the most part. This in, accordance with the previous research, indicates, that an individuals place in the society was determined by the family background. Yet, the density of the networks–depicting the importance of the family relations–changes with time. The graph of the 16th century is more sparse than the graph of the 17th century.

Probably one of the most telling part of the graphs were the isolated nodes or lack of thereof. In the graph depicting the 16th century, the isolated nodes were somewhat common, but the amount decreased in the 17th century graph. In some cases the nodes were of incomers to the Swedish political life, and therefore indicating social mobility. The councillors represented by these nodes usually led interesting life stories.

When it comes to the question of noble houses; the graphs did not form visually separate subgraphs or components, which would indicate diverge families or houses. Instead the well established and famous noble lines such as Gyllenstierna, De la Gardie and Sparre were present in the nodes with the most connections. In that case it seems that the family networks were connected to each other, and most prominent houses were integrally intervowen into the network of the councillors of the Realm.

Collecting and creating network data manually from scratch is a daunting and time consuming task. To make process of building the graphs from already existing data easier, I coded a small Python script to do the manual, error prone and dull work for me. Altogether, automatizing the process of extracting historical data is not that straightforward.

\begin{quote}
	Here is a truth: nobody wants to run your program. What they want is to get their work done, or play their game, or send their email\ldots The truth is that good code is invisible. It simply allows things to flow smoothly. Bad code is memorable. It interferes, makes people frustrated and angry.\footcite[prefix p. XVI]{python}
\end{quote}

Automatically processing data that is not designed and collected to be automatically processed is hard. Scripts does not have an understanding of context. The dataset and file names having small characters, such as dots and spaces, that are able to crash the script. These detrimental small characters have to be manually replaced form the dataset or to be excluded from the processed data. Also the slighest changes in the spelling is able to corrupt the whole model.

For instance, working on a similar task for the \textit{Shared Past, Different Interpretations} -project, I came across two different spellings of the same name in the dataset. This caused the script to create two nodes, of the one and same person, to the graph. To make matters worse, the person was a highly central figure on the network, and the conflicting nodes discretely broke the whole model. With the help of alert colleque, and by a lucky coinsidence, the error was spotted in time. Had not that happened, the graph would have remained inherently inaccurate. Usually it helps, if the programmer–or the working group–has at least conceptual understanding of the data. 

An important observation concerning the actual network analysis is; which statistics are usable in the context of historical social networks. For example, the graph density should tell us, how interwoven the network was. Instead, as the number of possible nodes increases so rapidly with the size of the network, the density of two networks, with even a little difference in size, is almost impossible to compare. Furthermore, the mathematical definition of dense network would be odd in the context of larger social networks. On the other hand, the far simpler parameters, such as, node degree or the ratio of isolated nodes from all nodes are way more depicting. 

While doing this work I made some simplifications on the data processing and model building. For instance, the calculated and reported numbers of edges are in fact the lower limit of the probable numbers. While limiting the time range of the graphs, some existing family links are excluded from the calculation, and, furthermore, some links were already excluded from the original dataset. This means that the actual networks were most likely slightly more linked. This deviation or uncertainty could be further quantified with the methods of Bayesian statistics.
 
As computational network analysis seems to be a working method in the study of early modern history, it could be implemented in the study of wider social groups. Reconstructing social networks with noble women or the clients of the noblemen could be a doable task in the future.

%TODO  parallel edges, ghosts