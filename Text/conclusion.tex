\section{Conclusion}
\textit{Can historical network analysis reveal new or unseen patterns in the affiliations between Swedish councillors of the Realm?} Using network analysis can indeed reveal new patterns on the given dataset. Though, implemented on the highly researched topic of the early modern nobility, it seems to confirm the already known. This is not inherently a bad thing after all. Creating and testing a (new) model with a strong theoretical background is not redundant work, it is an integral part of the ground research. 

\textit{How dense is the network (how linked the Council was in general)?} The graphs depicting the councillors' family ties in the early modern period are highly linked for the most part. This, in accordance with the previous research, indicates that an individual's place in the society was determined by the family background. Yet, the density of the networks–depicting the importance of the family relations–changes with time. The graph of the 16th century is more sparse than the graph of the 17th century.

\textit{Are there any isolated nodes (councillors who have no relatives in the Council)?} The isolated nodes or lack of thereof were probably one of the most revealing parts of the graphs. In the graph depicting the 16th century, the isolated nodes were somewhat common, but the amount decreased in the 17th century graph. In some cases the nodes were of incomers from abroad or of lower birth. Therefore they indicated both social and international mobility in the political sphere of early modern Sweden. The councillors represented by these nodes usually led interesting life stories.

\textit{Can the graph be visually divided into clear sub graphs (is there a certain groups or 'houses' of related councillors)?} The graphs did not form visually separate subgraphs or components, which would indicate individual families or houses. Instead, the members well established and famous noble lines, such as Gyllenstierna, De la Gardie and Sparre, were represented as the nodes with the most connections. Following the previous research, it seems that the family networks were generally connected to each other, and most prominent houses were integrally interwoven into the network of the councillors of the Realm.

Dividing the graph between 16th and 17th centuries further revealed the processes within lineages. Comparing the two graphs was able to show the extinction of the Bååt family and the simultaneous rise of the de la Gardie family. The ancient Bååt family was prominent in the graph depicting the 16th century, however in the graph of the 17th century only two of the councillors carries the surname Bååt. On the contrary, in the graph of the 16th century Pontus de la Gardie is a single representative of the family, whereas in the 17th century de la Gardie is amongst the most networked houses.

\textit{To what extent can pre-processing the dataset for the network analysis be automated with a script?} Collecting and creating network data manually from scratch is a daunting and time consuming task. To make process of building the graphs from already existing data easier, I coded a small Python script to do the manual, error prone and dull work for me. Altogether, automatizing the process of extracting historical data is not that straightforward, and making a working code demanding in itself. As computer scientist James R. Parker put it in his book: 

\begin{quote}
	Here is a truth: nobody wants to run your program. What they want is to get their work done, or play their game, or send their email\ldots The truth is that good code is invisible. It simply allows things to flow smoothly. Bad code is memorable. It interferes, makes people frustrated and angry.(\cite[prefix p. XVI]{python})
\end{quote}

Automatically processing data that is not designed and collected to be automatically processed is hard. Scripts do not have an understanding of context. If the dataset or file names contain minor characters, such as dots or spaces, the script may crash. These detrimental small characters have to be manually replaced from the dataset or to be excluded from the processed data. Also the slighest changes in the spelling is able to corrupt the whole model.

For instance, working on a similar task for the \textit{Shared Past, Different Interpretations} -project, I came across two different spellings of the same name in the dataset. This caused the script to create two nodes, of the one and same person to the graph. To make matters worse, the person was a highly central figure on the network, and the conflicting nodes discretely broke the whole model. With the help of alert colleague, and by a lucky coinsidence, the error was spotted in time. Had that not happened, the graph would have remained inherently inaccurate. Usually it helps, if the programmer–or the working group–has at least conceptual understanding of the data. 

\textit{What are the potential difficulties and pitfalls in the implementation and interpretation of historical network analysis in this specific dataset, and further in the field of early modern history?} An important observation concerning the actual network analysis is, which statistics are usable in the context of historical social networks. For example, the graph density should tell us how interwoven the network was. Instead, as the number of possible nodes increases so rapidly with the size of the network, the density of two networks, with even a little difference in size, is almost impossible to compare. Furthermore, the mathematical definition of dense network would be odd in the context of larger social networks. On the other hand, the far simpler parameters, such as node degree or the ratio of isolated nodes from all nodes, are way more depicting. 

While doing this work I made some simplifications on the data processing and model building. For instance, the calculated and reported numbers of edges are in fact the lower limit of the probable numbers. While limiting the time range of the graphs, some existing family links are excluded from the calculation, and, furthermore, some links were already excluded from the original dataset. This means that the actual networks were most likely slightly more linked. This deviation or uncertainty could be further quantified with the methods of Bayesian statistics.
 
As computational network analysis seems to be a working method in the study of early modern history, it could be implemented in the study of wider social groups. Reconstructing social networks with noble women or the clients of the noblemen could be a feasible project. Also the \textit{Swedish councillors of the Realm, 1523-1680} -dataset could be utilized in further research. For example creating a more complex statistical model on the age distributions of the councillors appointed by each monarch would be a doable and inspiring task in the future.

%TODO  parallel edges, ghosts