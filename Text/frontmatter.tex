% This front matter is from Birgitta Burger's tex template (v 1.1)
%
\begin{titlepage}
    \mbox{}\vfill
    \begin{center}
        {\bf\Large Game of Networks: Family Ties Within the Swedish Council of the Realm (1523-1680)}\\
        \vfill
        \begin{flushright}
            Heidi Suurkaulio\\[4pt]
            Master's thesis\\[4pt]%change this to term paper or whatever you like
            Historia, yleinen / history, general\\[4pt]
            Historian ja etnologian laitos / Department of history and ethnology\\[4pt]
            \today\\[4pt]
            Jyväskylän yliopisto / University of Jyväskylä
        \end{flushright}
    \end{center}
\end{titlepage}

\thispagestyle{empty}

{\bf\Large JYVÄSKYLÄN YLIOPISTO}

{\renewcommand{\arraystretch}{1.5}%local redefining of cell spacings
\begin{tabularx}{\textwidth}{|| X | X ||}
\hhline{|t:==:t|}
Tiedekunta -- Faculty		\newline		Humanistis-yhteiskuntatieteellinen tiedekunta	
&
Laitos -- Department		\newline		Historian ja etnologian laitos
\\\hhline{||--||}

\multicolumn{2}{|| p{\textwidth} ||}{
Tekijä -- Author 			\newline		Heidi Suurkaulio
}\\\hhline{||--||}

\multicolumn{2}{|| p{\textwidth} ||}{
Työn nimi -- Title 			\newline		Game of Networks: Family Ties Within the Swedish Council of the Realm (1523-1680)
}\\\hhline{||--||}

Oppiaine -- Subject			\newline 		Historia, yleinen
&
Ty\"on laji -- Level 		\newline		Pro gradu -tutkielma
\\\hhline{||--||}	

Aika -- Month and year		\newline		toukokuu 2025
&
Sivum\"a\"ar\"a -- Number of pages	\newline		TODO PAGES
\\\hhline{||--||}

\multicolumn{2}{|| p{\textwidth} ||}{
Tiivistelmä -- Abstract		\newline
Pro gradu -työni aiheena on Ruotsin valtaneuvosten väliset sukuverkostot aikavälillä 1523-1680. Työni yhdistää digitaalisia ihmistieteitä ja esimodernin ajan tutkimusta. Työn teoreettinen viitekehys ja menetlemä on tietokoneavusteinen verkostoanalyysi.
\newline
Työn aineistona on käytetty Marko Hakasen ja Ulla Koskisen kokoamaa \textit{Swedish Councillors of the Realm, 1523-1680} -tietokantaa. Aineistoon tallennetut valtaneuvosten sukulaisuussuhteet erotellaan automaattisesti omaksi tietorakenteekseen toteuttamallani Python-ohjelmalla. Varsinainen verkostoanalyysi, eli verkon visuaalinen esitys ja tunnuslukujen lasku, on tehty Gephi-työkalulla. Tämän lisäksi aineistosta lasketaan analyysia tukevia (tilastollisia) tunnuslukuja R-ohjelmointiympäristössä.
\newline
Kysymyksenasettelu työssäni liittyy sekä aineistoon että menetelmään. Verkostoanalyysi vahvistaa aikaisemman tutkimuksen antamaa kuvaa siitä, että Ruotsin ylhäisaateli oli varhaismodernilla ajalla verkostoitunutta niin suorien perimyssuhteiden kuin avioliittojenkin kautta. Erityisesti 1600-lukua verkosto on aikaisempaa 1500-luvun verkostoa tiheämpi. Valtaneuvosten verkostoista löytyi kuitenkin myös yksittäisiä henkilöitä, joilla ei ollut takanaan virallista sukuverkostoa. Sosiaalinen ja kansainvälinen liikkuvuus oli siis mahdollista, vaikkakin harvinaista.
\newline
Parhaimmillaan tietokoneavusteinen verkostoanalyysi tuottaa ennalta-arvaamatonta tietoa aineiston rakenteesta, ja tiedon käsittelyn automatisointi ohjelmoimalla nopeuttaa tutkimusprosessia ja vähentää ihmisen tekemiä (kirjoitus)virheitä. Aineiston käsittelyn automatisointi vaatii kuitenkin aineiston tuntemusta ja tarkkuutta, eivätkä kaikki verkostoanalyysissä käytetyt tunnusluvut sellaisenaan sovi historiallisen sosiaalisen verkoston analysointiin.
}\\\hhline{||--||}

\multicolumn{2}{|| p{\textwidth} ||}{
Asiasanat -- Keywords		\newline
Verkostoanalyysi, digihumanismi, varhaismoderni aika, Ruotsin valtakunta, valtaneuvosto, Gephi, Python, R, network analysis, computational history, early modern period, Swedish council of the Realm, Riksrådet
}\\\hhline{||--||}

\multicolumn{2}{|| p{\textwidth} ||}{
Säilytyspaikka -- Depository \newline
Jyväskylän yliopiston kirjasto
}\\\hhline{||--||}

\multicolumn{2}{|| p{\textwidth} ||}{
Muita tietoja -- Additional information
\newline
Git-sivusto: https://github.com/Heidi-Suurkaulio/mastersthesis/ 
}\\\hhline{|b:==:b|}
\end{tabularx}
}