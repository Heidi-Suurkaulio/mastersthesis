% This front matter is from Birgitta Burger's tex template (v 1.1)
%
\begin{titlepage}
    \mbox{}\vfill
    \begin{center}
        {\bf\Large Game of Networks: Family Ties Within the Swedish Council of the Realm (1523-1680)}\\
        \vfill
        \begin{flushright}
            Heidi Suurkaulio\\[4pt]
            Master's thesis\\[4pt]%change this to term paper or whatever you like
            Historia, yleinen / history, general\\[4pt]
            Historian ja etnologian laitos / Department of history and ethnology\\[4pt]
            \today\\[4pt]
            Jyväskylän yliopisto / University of Jyväskylä
        \end{flushright}
    \end{center}
\end{titlepage}

\thispagestyle{empty}

{\bf\Large JYVÄSKYLÄN YLIOPISTO}

{\renewcommand{\arraystretch}{1.5}%local redefining of cell spacings
\begin{tabularx}{\textwidth}{|| X | X ||}
\hhline{|t:==:t|}
Tiedekunta -- Faculty		\newline		Humanities	
&
Laitos -- Department		\newline		Department of History and Ethnology
\\\hhline{||--||}

\multicolumn{2}{|| p{\textwidth} ||}{
Tekijä -- Author 			\newline		Heidi Suurkaulio
}\\\hhline{||--||}

\multicolumn{2}{|| p{\textwidth} ||}{
Työn nimi -- Title 			\newline		Game of Networks: Family Ties Within the Swedish Council of the Realm (1523-1680)
}\\\hhline{||--||}

Oppiaine -- Subject			\newline 		History, general
&
Ty\"on laji -- Level 			\newline		Master's Thesis
\\\hhline{||--||}	

Aika -- Month and year		\newline		May 2025
&
Sivum\"a\"ar\"a -- Number of pages	\newline		TODO PAGES
\\\hhline{||--||}

\multicolumn{2}{|| p{\textwidth} ||}{
Tiivistelmä -- Abstract		\newline
Psychoacoustic research involves experimental settings. In these settings the experiment can influence the independent variables, for example stimulus intensities used for measuring detection thresholds.
\newline
\newline
The values of the independent variable can be determined \textit{adaptively} based on data gathered during the experiment. This can aid the researcher by 1) automating the selection process and 2) by selecting stimuli that maximize the amount of information gained. \textit{Bayesian adaptive estimation} uses principles derived from information theory and Bayesian statistics as a basis for the adaptive selection rules, and offers a well-defined theoretical background for adaptive testing.  
\newline
\newline
In this thesis I extend the existing Bayesian adaptive methods by showing how simulations can be used in the adaptive estimation framework beyond the settings in which is currently employed in. Through practical demonstrations to auditory filtering and reaction times I show how to implement the proposed method in in psychoacoustic research.
\newline
\newline
The main result of this thesis is that by implementing the simulation-based approach, adaptive Bayesian methods can be relatively easily applied to a wide range of problems not only in psychoacoustics but music research in general. This expands the applicability of these methods to many new applications. 
}\\\hhline{||--||}

\multicolumn{2}{|| p{\textwidth} ||}{
Asiasanat -- Keywords		\newline
Verkostoanalyysi, digihumanismi, varhaismoderni aika, Ruotsin valtakunta, valtaneuvosto, Gephi, Python, R, network analysis, computational history, early modern period, Swedish council of the Realm, Riksrådet
}\\\hhline{||--||}

\multicolumn{2}{|| p{\textwidth} ||}{
Säilytyspaikka -- Depository \newline
University of Jyväskylä
}\\\hhline{||--||}

\multicolumn{2}{|| p{\textwidth} ||}{
Muita tietoja -- Additional information
Git repository: 
}\\\hhline{|b:==:b|}
\end{tabularx}
}