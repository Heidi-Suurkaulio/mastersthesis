\documentclass[a4paper,12pt]{article}
\usepackage[english]{babel}
\usepackage[margin=1in]{geometry}
\usepackage{setspace}
\usepackage[backend=bibtex, style=chicago-authordate]{biblatex}
\addbibresource{mastersthesis.bib}
%whenever Islandic names
\usepackage[utf8]{inputenc}
\usepackage[T1]{fontenc}

\begin{document}
\begin{onehalfspace} %TODO move to the main tex component!
\section{Introduction}

Even though things like AI is hyped, it's important to go to the roots and basics of the computational methods. It's important to study this subject...

pre-modern period, explorative study

\subsection{Research questions}

My research questions are:
\begin{enumerate}
	\item Can social network analysis reveal some new or unseen patterns in the affiliations between Swedish Councilors of the Realm?
	\item TODO Can pre-processing datasets on creating network data be automatised?
	\item What are the potential difficulties and pitfalls in the implementation and interpretation of social network analysis in this specific dataset, and further in the field of pre-modern history?
\end{enumerate} 

\subsection{Previous research}
Historical network analysis can be understood as, in a certain degree established, but developing method. According to Finnish political historians Kimmo Elo and Olli Kleemola, the roots of historical network analysis are as far as in the late 19th century, yet, the modern appliance of the method is due the invention of computers, increase in the computing capacity and availability of user friendly network analysis software. In their article Elo and Kleemola estimate that historical network analysis has gained its popularity from somewhere in the late 2000's. It appears that Elo and Kleemola approaches historical network analysis as a predominantly digital or computational research method.\footcite[p. 415-417.]{eloAklee15} However, social network analysis, which is the base for historical network analysis and sometimes applied as such in the field of history, involves theorising, model building and empirical research focusing on the patterns formed inside the networks.\footcite[p. 22-24.]{Keats-R2007} And network analysis has been employed in the field of history before the turn of the millenia, previous to the era of intuitive software.\footcite[TODO check!]{AronssonEtA1999} That said, in the context of this work, (historical) network analysis will be treated primarily as a computer-aided method, similarily to the article of Elo and Kleemola. The further theory and practice will be covered in the section 2.

The international \textit{Historical Network Research Community} (HNR) was found in 2009. The community has grown over time, and nowadays HNR runs workshops, conferences, lectures, a Slack (chat) group, and publishes an open access journal, a newsletter and a research bibliography.\footcite{hnr} On the word of Kimmo Elo, historical network analysis is the most popular computational method amongst historians.\footcite[p. 22.]{elo16} 

Scanning the HNR research bibliography, it appears that historical network analysis has been applied by researchers and research teams from around the globe in variety of research topics. The topics vary from the social networks of Chinese gods to the technical implementation of historical network analysis, and to the historical study of reconnaissance during the Cold War.\footcites[p. 22.]{elo16}{hnrbib} Network analysis has also been utilized in the study of ruling elite and power in the pre-modern period.\footcite[See e. g. Ruth Ahnert's and Sebastian E. Ahnert's book \textit{Tudor Networks of Power} (2023) or Paul D Mclean's article \textit{Widening Access While Tightening Control: Office-Holding, Marriages, and Elite Consolidation in Early Modern Poland} (2004).]{JonVidarEt} Though, compared to the amount of all the historical research, computer-aided historical network analysis is still somewhat rarity.
 
In Finland, Kimmo Elo is one of the researchers highly profiled on the use of the historical network analysis. Among other things, he has co-authored two articles addressing the method in more explorative manner. The first of which is \textit{Verkostoanalyysi historiallisten aineistojen eksploratiivisena analyysimenetelmänä : esimerkkinä sotavalokuvat} written by Elo and Olli Kleemola. In the article they focus on the applicability of historical network analysis. As their data, they use German war propaganda pictures taken from Finland during the second world.\footcite{eloAklee15}

The another article is \textit{Networks of Revolutionary Workers: Socialist Red Women in Finland in 1918} written by Elo and political historian Tiina Lintunen. In this article the method of historical network analysis is applied on the connections between the women who participated to the Finnish civil war in 1918 on the side of the socialists also known as "red".\footcite[Almost the same article is found in Finnish in the \textit{Historiallinen Aikakauskirja} 116 (2/2018).]{LintunenAndElo2019} These articles share the exploratory perspective with this study, and for that, offer a point of reference. 

What comes to the literature discussing the Swedish Council of the Realm, it seems that as a significant administration its members and activities have been comparatively covered by previous research. For instance, the development and affairs of the Council as an institution are addressed in the works of historians such as Petri Karonen, Pentti Renvall and Kurt Agren.\footnote{See e. g. Petri Karonen: \textit{Pohjoinen Suurvalta} (2008) TODO check! or "\textit{The council of the realm and the quest for peace in Sweden, 1718-1721}" in \textit{Hopes and fears for the future in early modern Sweden, 1500-1850} (2009), Pentti Renvall "\textit{Keskitetyn hallintolaitoksen kehitys}" in \textit{Suomen kulttuurihistoria. II} (1934) or Kurt Agren "\textit{Rise and decline of an aristocracy: The Swedish social and political elite in the 17th century}" in the \textit{Scandinavian journal of history} (1976).} Additionally, short biographies of some members of the council of the realm can be found easily in the \textit{Biografiskt lexikon för Finland} (Biographical Dictionary of Finland).\footcite{blf} Those biographies include an assortment of notables found in the Councilors dataset, such as, Herman (Claesson) Fleming, Gabriel Bengtsson Oxenstierna (af Korsholm och Wasa) or Lorentz (Ernstsson) Creutz d.ä.\footcite{blf-list} Even so, the Council of the Realm, as a focal point, does still hold some unexamined parts.\footcite[p. 47-48.]{HakanenAKoskinen2017} 

Authors of the Councilors dataset, Marko Hakanen and Ulla Koskinen, have explained the dataset's background in their article \textit{The Gentle Art of Counselling Monarchs (1560-1655)}. In their study the council is approached through the concept of personal agency.\footcite{HakanenAKoskinen2017} In the article, Hakanen and Koskinen also mention some prior collection and utilisation of datasets on the councilors of the realm and their networks. Namely, Jan Samuleson has listed councillors and their affiliations from years 1523 – 1611, Kurt Ågren has collected councillors and their families from years 1602 – 1647, and Björn Asker made a similar collection from years 1640 – 1680. Yet, some of the datasets remain unpublished.\footcite[p. 48, 67 (cite 4).]{HakanenAKoskinen2017} The context of this study will be on the on the literature concerning the Council of the Realm, it will form the base for the applied network model and form the premises present in the data processing. 
 
Overall, it seems that amongst historians the pre-modern elite is collectively understood as a network, and the ties between the members have been in the scope of interest for some time now. Studying the Swedish Council of the Realm, or the family ties within the council, is nothing ground-breaking. However, the aim of this work is to join a still relatively uncommon method of computer-aided historical network analysis with the classic research topic, and to further test and develop the method in the context of historical research. 

\subsection{Sources}
\end{onehalfspace}
\end{document}