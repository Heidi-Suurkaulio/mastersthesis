\documentclass[a4paper,12pt]{article}
\usepackage[english]{babel}
\usepackage{setspace}
\usepackage[backend=bibtex, style=chicago-authordate]{biblatex}
\addbibresource{mastersthesis.bib}
%whenever Islandic names
\usepackage[utf8]{inputenc}
\usepackage[T1]{fontenc}

\begin{document}
\section{Introduction}

It's important to study this subject...

pre-modern period, explorative study

\subsection{Research questions}

My research questions are:
\begin{enumerate}
	\item Can social network analysis reveal some new or unseen patterns in the affiliations between Swedish Councilors of the Realm?
	\item TODO Can pre-processing datasets on creating network data be automatised?
	\item What are the potential difficulties and pitfalls in the implementation and interpretation of social network analysis in this specific dataset, and further in the field of pre-modern history?
\end{enumerate} 

\subsection{Previous research}
TODO
- Development of social network analysis. 
- Social media, influx of data and new methods.

In the field of history, network analysis can be understood as in a certain degree established but developing method. According to finnish political historians Kimmo Elo and Olli Kleemola, the roots of network analysis are as far as in the late 19th century, yet, the modern appliance of the method is due the invention of computers, increase in the computing capacity and availability of user friendly network analysis software. In their article Elo and Kleemola estimate that social network analysis has gained its popularity amongst historians from somewhere in the late 2000's.\footcite[p. 415-417.]{eloAklee15} It appears that Elo and Kleemola approaches network analysis as a predominantly digital or computational method. However, variations of network analysis has also been used in other methodological approaches, such as, prosopography or Actor-network-theory. That said, in the context of this work, network analysis will be treated as a digital method, similarily to the article of Elo and Kleemola. The further theory and implementation of (historical) network analysis will be covered in the section 2.

The international community for historical network analysis \textit{the Historical Network Research Community} (HNR) was found in 2009. The community has grown over time, and nowadays HNR runs workshops, conferences, lectures, an open access journal, a newsletter, a Slack (chat) group and a research bibliography.\footcite{hnr} On the word of Kimmo Elo, historical network analysis is the most popular computational method amongst historians.\footcite[p. 22.]{elo16} 

Scanning the HNR research bibliography, it appears that historical network analysis has been applied by researchers and research teams from around the globe in variety of research topics. The topics vary from the social networks of Chinese gods to the technical implementation of historical network analysis, and to the historical study of reconnaissance during the Cold War.\footcites[p. 22.]{elo16}{hnrbib} Network analysis has also been utilized in the study of ruling and power in the pre-modern period.\footcite[See e. g. Ruth Ahnert's and Sebastian E. Ahnert's book \textit{Tudor Networks of Power} (2023) or Paul D Mclean's article \textit{Widening Access While Tightening Control: Office-Holding, Marriages, and Elite Consolidation in Early Modern Poland} (2004).]{JonVidarEt}

In the context of Swedish or Finnish history... Kimmo Elo: Finnish civil war networks, 

Yet, compared to the amount of all the historical research, actual computer-aided social network analysis is somewhat rarity.

What comes to the literature discussing the Swedish Council of the Realm, it seems that as a significant administration its members and activities have been covered quite thoroughly. For instance, the development and affairs of the Council as an institution are addressed in the works of historians such as Petri Karonen, Pentti Renvall and Kurt Agren.\footnote{See e. g. Petri Karonen: \textit{Pohjoinen Suurvalta} (2008) or "\textit{The council of the realm and the quest for peace in Sweden, 1718-1721}" in \textit{Hopes and fears for the future in early modern Sweden, 1500-1850} (2009), Pentti Renvall "\textit{Keskitetyn hallintolaitoksen kehitys}" in \textit{Suomen kulttuurihistoria. II} (1934) or Kurt Agren "\textit{Rise and decline of an aristocracy: The Swedish social and political elite in the 17th century}" in the \textit{Scandinavian journal of history} (1976).} Furthermore, short biographies of some members of the council of the realm can be found easily in the \textit{Biografiskt lexikon för Finland} (Biographical Dictionary of Finland).\footcite{blf} Those biographies include an assortment of notables found in the Councilors dataset, such as, Herman (Claesson) Fleming, Gabriel Bengtsson Oxenstierna (af Korsholm och Wasa) or Lorentz (Ernstsson) Creutz d.ä.\footcite{blf-list}


 Marko Hakanen and Ulla Koskinen, who collected the Councilors dataset, have 
 
 Overall, it appears that amongst historians the pre-modern elite is collectively understood as a network, and the ties between the members have been in the scope of interest for some time now. Studying the Swedish Council of the Realm, or the family ties within the council, is nothing ground-breaking. However, the aim of this work is to join a still relatively uncommon method of computer-aided social network analysis with the classic research topic, and to further test and develop the method in the context of historical research. 

\subsection{Sources}

\end{document}