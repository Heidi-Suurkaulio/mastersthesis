\documentclass[a4paper,12pt]{article}
\usepackage[english]{babel}
\usepackage{setspace}
\usepackage[backend=bibtex, style=chicago-authordate]{biblatex}
\addbibresource{mastersthesis.bib}

\begin{document}
\section{Introduction}

It's important to study this subject...

pre-modern period, explorative study

\subsection{Research questions}

My research questions are:
\begin{enumerate}
	\item Can social network analysis reveal some new or unseen patterns in the affiliations between Swedish Councilors of the Realm?
	\item TODO Can pre-processing datasets on creating network data be automatised?
	\item What are the potential difficulties and pitfalls in the implementation and interpretation of social network analysis in this specific dataset, and further in the field of pre-modern history?
\end{enumerate} 

\subsection{Previous research}
In the field of social sciences, network analysis can be understood as a prevalent method.
TODO
- Development of social network analysis. 
- Social media, influx of data and new methods.
The actual method of network analysis will be explained in more detail on the section 2. 

Clearly less popular in practise, social network analysis is neither novelty in the field of history, with the international community for historical network analysis \textit{the Historical Network Research Community} (HNR) originating from year 2009. The community has grown over time, and nowadays it runs workshops, conferences, lectures, an open access journal, a newsletter, a Slack (chat) group and a research bibliography.\footcite{hnr} According to Finnish political historian \textit{Kimmo Elo}, historical network analysis is the most popular computational method amongst historians.\footcite[p. 22.]{elo16} 

Scanning the HNR research bibliography, it appears that historical network analysis has been applied by researchers and research teams from around the globe in variety of research topics. The topics vary from the social networks of Chinese gods to the technical implementation of historical network analysis, and to the historical study of reconnaissance during the Cold War.\footcites[p. 22.]{elo16}{hnrbib} Network analysis has also been utilized in the study of ruling and power in the pre-modern period, for example, in Ruth Ahnert's and Sebastian E. Ahnert's book \textit{Tudor Networks of Power} from 2023 or Paul D Mclean's article \textit{Widening Access While Tightening Control: Office-Holding, Marriages, and Elite Consolidation in Early Modern Poland} from as early as 2004. 

In the context of Swedish or Finnish history... Kimmo Elo: Finnish civil war networks, 

Yet, compared to the amount of all the historical research, actual computer-aided social network analysis is somewhat rarity.

What comes to the literature discussing the Swedish Council of the Realm, it seems that as a significant administration its members and activities have been covered quite thoroughly. For instance, short biographies of the members of the council of the realm can be found in the \textit{Biografiskt lexikon för Finland} (Biographical Dictionary of Finland).\footcite{blf} Those biographies include some notables found in the Councilors dataset, such as, \textit{Herman (Claesson) Fleming}, \textit{Gabriel Bengtsson Oxenstierna (af Korsholm och Wasa)} or \textit{Lorentz (Ernstsson) Creutz d.ä}.\footcite{blf-list} 


 Marko Hakanen and Ulla Koskinen, who collected the Councilors dataset, have 
 
 Overall, it appears that amongst historians the pre-modern elite is collectively understood as a network, and the ties between the members have been in the scope of interest for some time now. Studying the Swedish Council of the Realm, or the family ties within the council, is nothing ground-breaking. However, the aim of this work is to join a still relatively uncommon method of computer-aided social network analysis with the classic research topic, and to further test and develop the method in the context of historical research. 

\subsection{Sources}

\end{document}