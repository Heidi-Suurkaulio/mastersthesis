\documentclass[a4paper,12pt]{article}
\usepackage[english]{babel}
\usepackage[margin=1in]{geometry}
\usepackage{setspace}
\usepackage{graphicx} %for table
\usepackage[backend=bibtex, style=chicago-authordate]{biblatex}
\addbibresource{mastersthesis.bib}
%whenever Islandic names
\usepackage[utf8]{inputenc}
\usepackage[T1]{fontenc}

\begin{document}
\begin{onehalfspace} %TODO move to the main tex component!
\section{Introduction}
The aim of my master's thesis is to (re)construct and study a network of the family ties inside the Swedish Council of the Realm (Riksrådet) from the early 16th century to the late 17th century. The study will be conducted by creating a data structure and a visual representation of a graph depicting the family links. The work will be focused on roughly two main sectors: first of which is the actual analysis of the family links between the councilors, the second one is the assessment of the method of historical network analysis. 

In my thesis historical network analysis will be referred to as a computer-aided method, which links this work to the field of digital humanities. This study is also in the field of pre-modern history, because the timeframe of this study covers most of the 16th and 17th century. Methodologically, the study is quantitative study with an explorative approach.

Even though things like machine learning and artificial intelligence (AI) are particularly popular and to a certain degree hyped at the time of writing, it is important to assess and understand the fundamentals and basics of digital and computational methods. Obviously, it is easier to understand simpler models, and therefore ask important questions. Those questions are for instance: What are the premises for this model? What kind of data is suitable for this model? What kind of interpretations can be made from the results? What is the potential problem, how to fix or adjust the model if something goes wrong or unexpectedly?

The text will be divided in four sections. The introduction will present the premises of this work. As the method is a focal point of the study, it will be explained further in its own second section. The third section is about the practical implementation of the analysis and assesment of the results, and the last section collects everything together as a conclusion. 

\subsection{Research questions}
The research questions are:
\begin{enumerate}
	\item Can historical network analysis reveal new or unseen patterns in the affiliations between Swedish Councilors of the Realm? \begin{itemize}
		\item How dense is the network (how linked the council was in general)?
		\item Are there any isolated nodes (councilors who have no relatives in the council)?
		\item Can the graph be visually divided into clear subgraphs (is there a certain groups or 'houses' of related councilors)?
	\end{itemize}	
	\item What are the potential difficulties and pitfalls in the implementation and interpretation of historical network analysis in this specific dataset, and further in the field of pre-modern history? \begin{itemize}	
		\item To what extent can pre-processing the dataset for the network analysis be automated with a script?
	\end{itemize}
\end{enumerate} 

The source material for this study is the \textit{Swedish Councillors of the Realm, 1523-1680} dataset, which is also the timespan of the study. The period is relatively long and momentous in Swedish history, including many important events, such as, adoption of hereditary kingship (1544), the conflicts between the sons of Gustav Vasa (from 1560's to early 1600's), thrity years war (from 1618 to 1648) and queen Christina abdicating the throne (1654).\footcite[p. 8-9.]{personalAgency} These events and shifts obviously have had a signifficant impact on the ensemble and activities of the Swedish council of the realm. Yet, the focus of this study is (re)constructing a visual and computational network model of the family relations between the councilors, instead of the event-history. 

This study is conducted with a quantitative dataset instead of more traditional way of qualitative text sources. Therefore, the source criticism is done for the dataset as a whole. For example, by assessing the sources of the dataset, looking for possible human errors in the data and considering the original purpose and use of the dataset. 

The basis and context of this study will lay on the pre-existing literature concerning the Swedish Council of the Realm. Previous historical research will form the base for deciding the parameters for the network model (graph), and direct the choices for the data processing. These decisions include, for instance, whether or not draw the link between brothers if they are already connected to same father present in the graph. These decisions need to be based on the prior knowledge on the social relations during the pre-modern era. 

\subsection{Previous research}
Historical network analysis can be understood as, to a certain degree established, but developing method. According to Finnish political historians Kimmo Elo and Olli Kleemola, the roots of historical network analysis are as far as in the late 19th century, yet, the modern appliance of the method is due the invention of computers, increase in the computing capacity and availability of user friendly network analysis software. They estimate that historical network analysis has gained its popularity from somewhere in the late 2000's.\footcite[p. 415-417.]{eloAklee15} 

It appears that Elo and Kleemola approach historical network analysis as a predominantly digital or computational research method.\footcite[p. 415-417.]{eloAklee15} However, the definition is not that straightforward. Social network analysis, which is the basis for historical network analysis, involves theorising, model building and empirical research focusing on the patterns formed inside the networks.\footcite[p. 22-24.]{Keats-R2007} (Social) network analysis has been employed in the field of history before the turn of the millenia, previous to the era of intuitive software.\footcite[TODO check!]{AronssonEtA1999} So, the field of historical network analysis can be roughly divided in two approaches: one with more descriptive or theorising stance and the another that treats network analysis as a quantitative computer-aided method. In the context of this work, (historical) network analysis will be treated primarily as a computer-aided method, similarily to the article of Elo and Kleemola, therefore focusing mainly on the previous research with computational approach. The further theory and practice will be covered in the section 2.

The international \textit{Historical Network Research Community} (HNR) was found in 2009. The community has grown over time, and nowadays HNR runs workshops, conferences, lectures and a Slack (chat) group, and publishes an open access journal, a newsletter and a research bibliography.\footcite{hnr} On the word of Kimmo Elo, historical network analysis has been the most popular computational method amongst historians.\footcite[p. 22.]{elo16} 

Scanning the HNR research bibliography, it appears that historical network analysis has been applied by researchers and research teams from around the globe in variety of research topics. The topics vary from the social networks of Chinese gods to the technical implementation of historical network analysis, and to the historical study of reconnaissance during the Cold War.\footcites[p. 22.]{elo16}{hnrbib} More relevant for this study, network analysis has been utilized in the study of ruling elite and power in the pre-modern period.\footcite[See e. g. Ruth Ahnert's and Sebastian E. Ahnert's book \textit{Tudor Networks of Power} (2023) or Paul D Mclean's article \textit{Widening Access While Tightening Control: Office-Holding, Marriages, and Elite Consolidation in Early Modern Poland} (2004).]{JonVidarEt} 
 
In Finland, Kimmo Elo is one of the researchers highly profiled on the use of the historical network analysis. Among other things, he has co-authored two articles addressing the method in more explorative manner. The first article is "\textit{Verkostoanalyysi historiallisten aineistojen eksploratiivisena analyysimenetelmänä : esimerkkinä sotavalokuvat}" written by Elo and Olli Kleemola. In the article they focus mainly on the applicability of historical network analysis. As their data, they use German war propaganda pictures taken from Finland during the second world war.\footcite{eloAklee15}

The another article is "\textit{Networks of Revolutionary Workers: Socialist Red Women in Finland in 1918}" written by Elo and political historian Tiina Lintunen. In this article the method of historical network analysis is applied on the connections between the women who participated to the Finnish civil war in 1918 on the side of the socialists also known as "reds".\footcite[Almost the same article is found in Finnish in the \textit{Historiallinen Aikakauskirja} 116 (2/2018).]{LintunenAndElo2019} Both of these articles share the exploratory perspective with this study, and therefore, offer a point of reference. 

When it comes to the literature discussing the Swedish Council of the Realm, it seems that as a significant administration its members and activities have been to some extent covered by previous research. For instance, the development and affairs of the council as an institution are addressed in the works of historians such as Petri Karonen, Pentti Renvall and Kurt Agren.\footnote{See e. g. Petri Karonen: \textit{Pohjoinen Suurvalta} (2008) TODO check! or "\textit{The council of the realm and the quest for peace in Sweden, 1718-1721}" in \textit{Hopes and fears for the future in early modern Sweden, 1500-1850} (2009), Pentti Renvall "\textit{Keskitetyn hallintolaitoksen kehitys}" in \textit{Suomen kulttuurihistoria. II} (1934) or Kurt Agren "\textit{Rise and decline of an aristocracy: The Swedish social and political elite in the 17th century}" in the \textit{Scandinavian journal of history} (1976).} Additionally, short biographies of some members of the council can be found easily in the \textit{Biografiskt lexikon för Finland} (Biographical Dictionary of Finland).\footcite{blf} Those biographies include an assortment of notables found in the Councilors dataset, such as, Herman (Claesson) Fleming, Gabriel Bengtsson Oxenstierna (af Korsholm och Wasa) or Lorentz (Ernstsson) Creutz d.ä.\footcite{blf-list} Even so, the Council of the Realm has not been examined thoroughly down to the last man. And based on historians Marko Hakanen and Ulla Koskinen, the council as a focal point, does still hold some uncovered parts.\footcite[p. 47-48.]{HakanenAKoskinen2017} 

Authors of the Councilors dataset, Hakanen and Koskinen, have explained the dataset's background in their article \textit{The Gentle Art of Counseling Monarchs (1560-1655)}. In their study the council is approached through the concept of personal agency.\footcite{HakanenAKoskinen2017} In the article, Hakanen and Koskinen also mention some prior collection and utilisation of datasets on the study of said councilors and their networks. Namely, Jan Samuleson has listed councilors and their affiliations from years 1523 – 1611, Kurt Ågren has collected councilors and their families from years 1602 – 1647, and Björn Asker made a similar collection from years 1640 – 1680. Unfortunately, some of the datasets remain unpublished.\footcite[p. 48, 67 (cite 4).]{HakanenAKoskinen2017} 

All in all, computer-aided historical network analysis is somewhat rare compared to the traditional methods of historiography. Nevertheless, it also seems that the pre-modern elite is collectively understood as a network amongst historians, and the ties between the members of nobility have been in the scope of interest for some time now. Which makes applying the computer-aided network analysis relevant. The aim of this work is to join the rather uncommon method of historical network analysis with the classic research topic, and to further explore and develop the method in the context of historical research.

\subsection{the Council of the Realm}

\subsection{Sources}
Since this work is conducted with preliminarily collected dataset, this work can be categorised as secondary analysis. Secondary analysis meaning the re-analysing the data with new research questions or approaches, while primary analysis involves the collection of the data. Secondary analysis can alse be discern from meta analysis, which means comparing multiple previous researches (usally with quantitative methods) to create a synthesis on a certain question.\footcite[p. 4-5.]{meta-analysis} 

On the contrary in their articles Kimmo Elo and Olli Kleemola or Elo and Tiina Lintunen applies a network analysis on their own primary datasets.\footcites{eloAklee15}{LintunenAndElo2019} However, in the case of this work the benefit of doing secondary analysis is that the focus can be on the implementation and assessment of the method. Furthermore, the existing dataset will be automatically and manually re-examined for possible errors in the process, as will be discussed soon. 

As mentioned earlier, this work is based on the \textit{Swedish Councilors of the Realm, 1523-1680} -dataset authored by Marko Hakanen and Ulla Koskinen. The dataset was published in 2017 and can be found in Digital repository of University of Jyväskylä under the license CC BY 4.0. The dataset was collected as a part of the research conducted for the anthology \textit{Personal Agency at the Swedish Age of Greatness 1560–1720}.

The dataset consists of information from 257 Swedish councilor of the realm. Each councilor has the following feature attributes: date of birth, year of death, year, date and age of appointment, noble rank, spouse(s) along with father's spouse and the individual's family links between other councilors. The councilors are identified with their full name and id number.\footcites[p. 48.]{HakanenAKoskinen2017}{councilorsDS}

\begin{table}[h]
	\centering
	\caption{Example rows of the dataset: Gyllenhorn, Joen Olsson and Natt och Dag, Måns Johansson (\cite{councilorsDS})} 
	\resizebox{\textwidth}{!}{%
		\begin{tabular}{|c | c | c | c | c | c | c | c | c | c|}
			\hline
			Name & No. & D.O.B. & † & Appointed & Date & Age & Noble rank & Family members in the council & Spouse(s) / Father of Spouse / Date of Marriage \\ 
			\hline
			Gyllenhorn, Joen Olsson & 82 & & 1556 & 1529 & 00.6. & & Uradel (Ancient Nobility) & Son-in-law 44 & Karin Bese/Nils Nilsson Bese/1529 \\
			\hline
			Natt och Dag, Måns Johansson & 142 & 1498 & 1555 & 1529 & 00.6. & 31 & Uradel (Ancient Nobility) & & Barbro Eriksdotter/Erik Turesson Bielke/ probably 27.6.1524 \\
			\hline
	\end{tabular}%
}
\end{table}
 
Hakanen and Koskinen have compiled the data from secondary sources such as biographical registers and databases, biography collections, lineage databases and research literature. The dataset's sources are listed in the dataset and in the article written by Hakanen and Koskinen, those include for instance Nordic Family Book (Nordisk familjebok), National Biography of Finland (Finland's nationalbiografi, Kansallisbiografia), genealogies of old Swedish aristocratic families (Äldre svenska frälsesläkter) and further refereed literature.\footcites[p. 48, 76]{HakanenAKoskinen2017}{councilorsDS} As reported by Hakanen and Koskinen the dataset is collected using methodologies of collective biography, new prosopography and source criticism.\footcite[p. 48.]{HakanenAKoskinen2017}
  
Even though the dataset can be assessed reliable and generally accurate, there is the general problem of some missing data. As seen in the example of table 1, some of the councilors have missing attributes such as date of birth and therefore age of appointment. Missing birthdays may simply be due the fact that in the pre-modern era there was no standard to record birthdays, and sometimes even the birthdays and ages of royal family members, whose lives have been mainly well documented, must be assessed by historians.(TODO source) However, the focus of this work is not in the ages of the councilors but in their affiliations, so, the more relevant question is whether or not there are missing family links. 

(TODO write script and do some checking)

As the relatively large dataset is compiled by humans, it leaves some room for typos and errors. While producing the first experimental graph of the dataset, there were found some empty data points These "ghosts" were nodes with only id number and one or two links to the other councilors. All in all there were four "ghosts" with the id numbers 147 (linked to 18 and 152), 215 (linked to 217), 249 (linked to 269) and 254 (linked to 94). With the help of Marko Hakanen it was resolved that those "ghost" nodes were data points removed from the dataset as the authors found out they have not been official councilors, however, some references to their id numbers had been left to the dataset by accident. The "ghost" nodes will be removed from the latter graphs.

\printbibliography
%\subsection{Sources}
%excluding women and lower classes
\end{onehalfspace}
\end{document}