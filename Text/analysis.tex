\section{Family ties between 1520 and 1680}
During the timeperiod between 1520 and 1680 there were 257 men active in the Council of the Realm. As discussed earlier in subchapter \ref{councilofrealm} the council consisted mostly of men with noble background. To quantify this: according to the dataset 236 of the 257 councillors (91.8\%) were part of the ancient nobility "uradel", and only approximately 5\% were of unknown or ignoble background, most of the time bishops or other clerics.  

\begin{table}
	\caption{Absolute amount in different ranks}
	\centering
	\begin{tabular}{cccccc}
		\hline
		Commoner & Ennobled & Estate unknown & Unknown & Ancient nobility & \\
		\hline
		5 & 8 & 7 & 1 & 236 & = 257 \\
		\hline
	\end{tabular}
	\caption{Proportional amount in different ranks}
	\centering
	\begin{tabular}{cccccc}
		\hline
	    Commoner & Ennobled & Estate unknown & Unknown & Ancient nobility & \\
	    \hline
	    1.9 & 3.1 & 2.7 & 0.3 & 91.8 & $\approx$ \% \\
	\end{tabular}
\end{table}

The kingdom of Sweden was ruled by nine monarchs: Gustavus Vasa (1523-1560), Eric XIV (1560-1568), Johan III (1568-1592), Sigismund (1592-1599), Charles IX (1599-1611), Gustavus Adolphus (1611-1632), Christina (1644-1654), Carl X Gustav (1654-1660) and Carles XI (1672-1697), and two regnants from 1632 to 1654 and from 1660 to 1672 during the timeperiod. The amount of councillors appointed by each monarch or regnant varied from none to 56. Yet, 13 councillors were present before the reign of Gustavus Vasa.

The monarchs who appointed the most councillors were Gustavus Vasa (56) and queen Christina (45). The large number of councillors appointed by Gustavus Vasa may be explained by the sheer length of his reign. He ruled Sweden for 37 years, whereas all of the other monachs ruled less than 20 years. Also the religious Reformation was perfromed during his reign. The bishops and clerics that lost their position in the Council were replaced by noblement chosen by the king.\footcite[TODO]{pSuurvalta} As queen Christina ruled only for ten years, the exceptionally high number on her reign is interesting. Therefore, her reign will be discussed in greater detail in subchapter \ref{christina}. 

However, the king who appointed no new councillors was the son of Johan III: Sigismund (). 
Why king Sigismud did not appoint any new councillors? 

\begin{figure}
	\includegraphics[scale=0.6]{councillorspermonarch.png}
	\centering
	\caption[Number of councillors appointed by each ruler between 1523-1680] {Number of councillors appointed by each monarch or regnant from 1523 to 1680. (\cite{councillorsDS})} 
	\centering
\end{figure}

All of the councillors active between 1520 and 1680 are represented in a graph of 257 nodes and 362 edges, with self loops and parallel edges removed. The number of edges exceeds the number of nodes which means that theoretically all of the nodes could be linked to one another. Although, the edges are not distributed that way. Some of the nodes are linked to more than one other node, but some isolated nodes are indeed present. The average degree of the graph is 2.817 $\approx$ 2.8 meaning that councillors typically had 2 or 3 direct relations to another members of the Council.

The graph density is 0.011. In the mathematical definition the graph is sparse (scale from 0 to 1). However, interpretation in practical sense is not that straightforward. Having a completely dense social network, meaning that every node is directly linked to each other, is almost impossible. In this context it would mean, that each councillor is a direct relative through blood or marriage to each other, which would be weird to say at least. Further, taking the dimension of time into account, the mathematical interpretation would be even more nonsensical; how a nobleman appointed to the council in the 1670's could be directly linked to a bishop died in 1530's. 

The question wheither or not the councillors were higly linked to each other is more of a qualitative one. In the context of early modern institution, how we define a higly linked or even nepotistical institution? Probably the graph density has more explanatory value as a parameter to be compared between graphs collected from other contemporary communities.

Even so, the proportion of the nodes with at least one connection in the graph is $224/257=0.8715...$ Which means that between 1520 and 1680 the probability of a councillor being related to someone else in the Council is $\approx$ 87\%. In that sense the councillors can be considered highly networked.

Another interesting finding are the nodes with the highest and lowest degree. Maybe the most eclectic and fascinating group is the isolated nodes. There are 33 isolated nodes meaning that 33 councillors did not have any family ties within the Council. Some of them are members of the nobility with family ties to councillors active prior the beginning year of the dataset, so, the connections are excluded from this graph. These are for example: Ture Bengtsson Lilliesparre (127) or Nils Olofsson Vinge (242). Yet, some of the isolated nodes represent bishops and clerics like Magnus Sommar (186) or Ingemar Petri (162) present in the Council before the religious Reformation. 

Besides that, the isolated nodes also reveals something about the international relationships during the timeperiod. For example a teacher from France Dionysius Beurraeus (12) or a Scottish baron Robert Douglas (58) can be found. Also the famous German born chancellor Conrad von Pyhy (168) is represented as an isolated node. During his stay in Sweden von Pyhy had a significant impact on the politics of Gustavus Vasa.\footcite[p. 81-83.]{pSuurvalta} This means that as a merited foreigner one could make their way up to the Swedish Council of the Realm.

On the contrary, the nodes with the highest degree ($\geq$ 5) are of councillors who are part of the old well established aristocratic families like De la Gardie, Oxentierna, Gyllenstierna, Bielke, Stenbock, Sparre and Ribbing. Also Finnish families of Horn and Kurck can be found.\footnote{TODO} That in itself implies to the previously known fact that the aristocratic families had a remarkable stance in the negotiations and politics of the Swedish kingdom.

To focus more on the temporal dimension of the graph, I decided to divide the dataset to half. As seen in Figure \ref{fig:peryear} the most intuitive point to do the division is between years 1600 and 1601. In year 1601 duke Charles had won the civil war against king Sigismund and he practically decimated the Council of the Realm while eliminating the noblemen loyal to the former king. However, by 1602 he had to appoint 15 new councillors in order to manage the growing number of tasks in diplomacy and administration.\footcite[TODO]{pSuurvalta} By coinsidence, year 1601 also divides the timerange of the dataset in half. The firs graph depicts the network prior year 1600 and the latter one post year 1601, so, both of these represents the family ties accumulated approximately for 80 years.

\subsection{Prior 1600}
The graph depicts the family links within the Council during most of the 16th century. It consist of 111 nodes and 92 edges. In this case the number of edges subceeds the number of nodes meaning that at least some isolated nodes must be present. 

The average degree in this graph is 1.658 $\approx$ 1.7, meaning that typically councillors were directly related to one or two other members. The number is smaller than in the larger graph. However, it can be partly explained by the shorter time range, simply the family networks had not have enough time to accumulate. Thereafter, the closest comparison is the graph of councillors family ties post 1601, which will be discussed later.

The graph density is 0.015, a slightly higher than in the larger graph. This is most likely due the sheer difference in size between the two graphs. The amount of nodes having at least one connection is $85/111 = 0.765...$ meaning that the probability of councillors to be linked at least one other member is $\approx 77\%$. That is 10 percentage points smaller than in the larger graph. Even though this prior year 1600 graph is not directly comparable with the larger graph, it indicates that the family links in the 16th century were more sparse.

The amount of isolated nodes is significant: 26–as in the larger graph it was 33. The fact that, all of the bishops and clerics must be present in this graph, partly explains this. Limiting the time range also breaks some of the existing links, so for example, Pontus De la Gardie seems to be without any connections in that graph, even though his descendands were remarkably woven into the networks of Swedish aristocracy. Taking these factors into account, still the relatively high number of isolated nodes may suggest the relatively high number of incomers and larger social mobility in general.

The nodes with highest degree ($\geq$ 4) were of the remarkable noble families such as Bielke and Stenbock, however the Bååt family is more apparent in this graph than in the larger one or any of the later ones (TODO check). Did the house loose its standing somehow, or did they blend with the other noble families?

\subsubsection{the Last Man Standing: Nils Gyllenstierna}
In fact the last councillor to be executed in Sweden was in 1605.%TODO 

\subsection{Post Duke Charles' revenge}
Practically, this graph begins from the year 1602 when duke Charles had to appoint 15 new councillors and the Council of the Realm was re-established. The graph consists of 146 nodes and 184 edges, now, the number of edges exceeds the number of nodes. After 1601 (or 1602) the average degree of node is 2.251 $\approx$ 2.3, larger than the 1.7 prior 1600. And, visually the graph seems more dense. 

The number of isolated nodes is significantly lower: 16, compared to the 26 prior 1600 or the 33 in the whole time period. Again it still must be remembered that limiting the time range cuts some family ties present in the dataset. Subsequently, the proportion of nodes with at least one edge is $130/146=0.890...$ ,thereby, the probability of councillor having at least one direct relation in the Council was now $\approx 89\%$. That is 12 precentage points greater than the 77\% in the 16th century, and two presentage points greater then the 87\% from the whole time period.

Also the density of this graph is the largest: 0.017 compared to the 0.015 of prior 1600 and the 0.011 from the whole time period. The scale of the larger graph makes it uncomparable to the smaller graphs. As discussed in the subchapter \ref{network} the number of possible edges in the biggest graph is 32 896. In the graph prior 1600 the number is 6 105, and in this; post 1601 graph it is 10 585. Even the size not working on favour of this post 1601 graph, it still has higher density compared to the prior 1600 graph. With these numbers it can be argumented that the post 1601 graph is more dense than the one prior 1600. 

What the isolated nodes can tell us?

All in all, 17th century has been called in the previous literature "the century of the nobility", and the presented data seems to confirm the trend.\footnote{TODO} It seems that the family affiliations became more prominent in the Council of the Realm, and important in general, during the 17th century. Similarily, the amount of incomers decreased, which implies the slackening of social mobility.

\subsection{In the Court of Queen Christina}
\label{christina}
Queen Christina was the daughter of the king Gustavus Adolphus and queen ...

As queen Christina's reign lasted for ten years the graph is not comparable to the other larger graphs, despite that, it is interesting in its own right. 

Rosenhane who gave the Hortus Regius book

For example king Gustavus Adolpus' illegitimate son Gustaf Gustafsson af Vasaborg (241) can be found on the Council. 
\begin{figure}
	\includegraphics[width=\linewidth]{councillors_1644-1654.png}
	\caption[Councillors appointed by queen Christina]{A graph of councillors appointed by queen Christina between 1644 and 1654.(\cite{councillorsDS})} 
	\centering
\end{figure}

